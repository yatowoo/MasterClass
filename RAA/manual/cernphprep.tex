\documentclass{cernphprep} 
\usepackage{cernunits,cernchemsym,heppennames2}
\usepackage{cite,fancyvrb}
\providecommand*\eg{e.g.,\xspace}
\providecommand*\ie{i.e.,\xspace}
\providecommand*\etc{etc.\@\xspace}
\providecommand*\etal{\emph{et al.}}
\providecommand*\cf{\emph{cf.\xspace}}
\begin{document}
\begin{titlepage}
\PHnumber{2010--xxx}
\PHdate{DD Month 2010}
\EXPnumber{IT-UDS-HUS-001}
\EXPdate{30 June 2010}
\DEFCOL{CDS-Library}
\title{Preparing a CERN PH preprint}
\author{Michel Goossens/IT-UDS\thanks{\texttt{Email:michel.goossens@cern.ch}}}
\begin{abstract}
This document contains instructions for authors who want to prepare a
CERN PH Preprint using the \texttt{cernphprep} \LaTeX{} class.  It
first discusses mandatory material that has to go on the title
page. Then it gives a brief overview of \LaTeX's structural elements
and their use. Suggestions for using correct spelling and grammar
follow, with particular attention to typographical rules for scientific
texts.  The appendix explains which packages are available to make
your life easier for typesetting your texts in a convenient and
coherent way.
\end{abstract}
\end{titlepage}
 
\section{General introductory remarks}
 
Authors are invited to observe the suggestions described in the
present document in order to ensure consistency and uniformity in the
style and layout of CERN PH preprints.

Although the typesetting rules that are described hare are of a
general nature, the present document explains their use with the
\texttt{cernphprep} \LaTeX{} class. This class file becomes
automatically available when running with the latest \LaTeX{} on
CERN's central \texttt{lxplus} system\footnote
%
    {See \url{http://cern.ch/XML/textproc/texlivelinux2009.html} for
     an explanation on modifying your \texttt{PATH} environment variable
     correctly.}.
%
 
As the text frame is defined as $16\Ucm \times 24\Ucm$ (\ie a A4 sized
paper has $3\Ucm$ margins at the top and at the bottom, and $3\Ucm$
at the inner and $2\Ucm$ at the outer edge), please take care to
remain inside these limits. This is especially true for figures and
tabular content.

Note that it is the responsibility of the authors to obtain permission
from the copyright holder if material taken from other sources is
included in the source submitted as a CERN PH preprint.

\section{The title page}

Your CERN PH preprint \textbf{must always} begin with a title page. The
information contained on the title page is specified inside a
\texttt{titlepage} environment.

The minimal mandatory information that a title page must contain is
the following (as, \eg for the present document).
\begin{Verbatim}[numbers=left,fontsize=\small]
\documentclass{cernphprep} 
\usepackage{cite,fancyvrb}
\begin{document}
\begin{titlepage}
\PHnumber{2010--xxx}
\PHdate{DD Month 2010}
\EXPnumber{IT-UDS-HUS-001}
\EXPdate{23 June 2010}
\DEFCOL{CDS-Library}
\title{Preparing a CERN PH preprint}
\author{Michel Goossens/IT-UDS\thanks{\texttt{Email:michel.goossens@cern.ch}}}
\begin{abstract}
This document contains instructions for authors who want to prepare
a CERN PH Preprint using the \texttt{cernphprep} \LaTeX{} class.
...
\end{abstract}
\end{titlepage}
\end{Verbatim}
On line 2 you see how you can specify supplementary packages to use in
your document (here, the \texttt{cite} and \texttt{fancyvrb} packages
are loaded). The commands \verb!\PHnumber! and \verb!\PHdate!  (lines
5 and 6) specify the number and the date that have been assigned by
the PH Departmental Office to your CERN PH preprint. You can add a
number and date for your experiment by using the \verb!\EXPnumber! and
\verb!\EXPdate! commands (lines 7 and 8). By default these are
empty. Then follow (lines 9 and 10) the title and the author(s)
(inside, resp., \verb!\title! and \verb!\author!  commands).  Finally,
the \texttt{abstract} environment (lines 11--15) lets you give a short
description of the content of the preprint. The command
\verb!\CERNCopyright! contains the copyright message that can appear
at the bottom of the front page (default: no message).

\subsection{Controling the running titles}

For the body of the text the running titles will show the (beginning
of the) title on the odd (right-hand) pages and the authors on the
even (left-hand) pages. 

If the way the authors or the title which are printed in the running
titles is not what you want, you can control the typeset text by
specifying what you want inside the \texttt{titlepage} environment as
argument to a \verb!\ShortAuthor! or \verb!\Shorttitle! command,
respectively.

\subsection{Special setup for the LHC experiments}

For the LHC experiments, the experiment logo can also be
printed. Experiments can contact me to have their logo registered so
that their experiment's name is recognized as an option by the class
file.

\begin{Verbatim}[numbers=left,fontsize=\small]
\documentclass[CMS]{cernphprep}
\begin{document}%
\begin{titlepage}
\EXPnumber{QCD-09-010}
\PHnumber{...}
\PHdate{...}
\title{Full title for paper...}
\Collaboration{The CMS Collaboration%
         \thanks{See Appendix~\ref{app:collab} for the list of collaboration 
                      members}}
\ShortAuthor{The CMS Collaboration}
\ShortTitle{Short title}
\begin{abstract}
  ...
\end{abstract}
\end{titlepage}
\end{Verbatim}
The \texttt{[CMS]} option specified on the \verb!\documentclass!
command (line~1) signals that we are dealing with a preprint prepared
by the CMS experiment. Thus the CMS logo will also appear on the title
page.  We note the presence of the \verb!\Collaboration!  command,
which substitutes for the \verb!\author! command (lines~8--10) with a
reference to the appendix (via the \verb!\thanks! command) which
contains the names of the many authors of that paper. Note also the
\verb!\ShortAuthor! and \verb!\ShortTitle! commands define the text
for the running headers.

\subsection{Alternative ways of specifying author lists}

For experiments with a smaller number of authors, the
\texttt{cernphprep} class offers two ways of specifying the authors
and affiliations on the title page, as shown in Fig.~\ref{fig:title}.
The \LaTeX{} sources for the layouts shown at the left and right in
Fig.~\ref{fig:title} correspond to the example files
\texttt{cernphprepexa.tex} and \texttt{cernphprepexax.tex}, respectively.

\begin{figure}
\centering
\fbox{\includegraphics[width=.46\linewidth]{cernphprepexa1.pdf}}
\hfill
\fbox{\includegraphics[width=.46\linewidth]{cernphprepexax1.pdf}}
\caption{Two possible layouts of authors and affiliations on the title page}
\label{fig:title}
\end{figure}

\section{The document body}

\subsection{Sectioning commands and paragraphs}
\label{sec:sections}

The standard \LaTeX{} commands \verb!\section! (level 1),
\verb!\subsection! (level 2), \verb!\subsubsection! (level 3),
\verb!\paragraph! (level 4), should be used for headings. 

To start a new paragraph it is sufficient to insert a blank line,
hence it is unnecessary to include \verb!\par! commands in your
\LaTeX{} source.

Please minimize explicit line break (\verb!\newline!, \verb!\\!) or
page break (\verb!\newpage!) commands and use them only when
\emph{absolutely} necessary and when the document is in its final
state.

\subsection{Equations}
\label{sec:equations}

An unnumbered single formula is delimited by \verb!\[! and \verb!\]!
(alternatively a \texttt{displaymath} environment can be used), while a
numbered equation is generated with the \texttt{equation} environment.
You should \textbf{never} use the \texttt{\$\$} construct.  

For cross-referencing equations use the \LaTeX{} commands \verb!\label!
and \verb!\ref!, as explained in Section~\ref{sec:crossref}.
A simple example, with a cross-reference to a formula, follows.
\begin{Verbatim}[numbers=left,fontsize=\small]
Einstein has expressed the relation between energy $E$ 
and mass $m$ in his famous formula~(\ref{eq:einstein}):
\begin{equation}
E=mc^2~.
\label{eq:einstein}
\end{equation}
\end{Verbatim}
When typeset, this gives:

Einstein has expressed the relation between energy $E$ and mass $m$ in
his famous formula~(\ref{eq:einstein}):
\begin{equation}
E=mc^2~.
\label{eq:einstein}
\end{equation}

Equations should be treated as part of the text, and therefore
punctuated (with a space, \eg \verb!~!, between the end of the
equation and the punctuation mark, as shown in
Eq.~(\ref{eq:einstein}).  Equations are numbered consecutively
throughout the report.

The CERN \LaTeX{} classes automatically load the \texttt{amsmath}
class, which offers a large choice of constructs for typesetting
mathematics~\cite{bib:voss2005}. See also Appendix~\ref{app:amsmath}
for some hints about using the \texttt{amsmath} extensions in an
optimal way.

\subsection{Figures}
\label{sec:figures}

Figures should be prepared electronically. The image should be of high
quality when printed (in black and white). In particular, all lines
should be heavy enough to be visible and text large enough to be
legible when the figure is printed \emph{at the final size}. If using
colour, the figure should print clearly in greyscale\footnote
%
  {Beware of faint colours, such as yellow or light blue or red, which
   are barely visible when reproduced as level of grey.}.

When running \texttt{latex} EPS (Encapsulated PostScript) files are to
be provided, while when running \texttt{pdflatex} PDF, PNG or JPEG
formats can be used.

As already mentioned in the introduction, before using material such
as illustrations taken from other sources, you must obtain permission
from the copyright holder.

All figures must remain within the page area ($16\Ucm \times 24\Ucm$),
where, if necessary, the page may be turned $90^\circ$ to accommodate
the figure. When this is done, the caption must be oriented in the
same way as the figure, and no other text may appear on that page.
The bottom of the turned illustrations should be at the right-hand
side of the page.

\subsubsection{Including your figure}

Figures are included with the \texttt{figure} environment. In the
example that follows we include an EPS graphics file
(\texttt{myfig.eps}, where the extension \texttt{.eps} does not need
to be specified since it is the default) with the
\verb!\includegraphics! command, which is defined in the
\texttt{graphicx} package that is loaded by default by the CERN
\LaTeX{} classes.

The figure caption (specified as argument of the \verb!\caption!
command), must \emph{follow} the figure body, and should be brief. No
full stop is necessary unless the caption is more that one sentence
long, in which case full punctuation should be used.

\subsubsection{References to figures}

\LaTeX's cross-reference mechanism can be used to refer to figures.  A
figure reference is defined with a \verb!\label! command, which must
come in the source \emph{after} the \verb!\caption! command that
contains the description of the figure content.

Figures must be referenced in the text in consecutive numerical order
with the help of the \verb!\ref! command. The following are examples
of references to figures and how to produce them.
\begin{Itemize}
\item `Fig. 3' produced by, \eg \verb!Fig.~\ref{fig:myfig}!,
\item `Figs. 3--5' produced by, \eg
       \verb!Figs.~\ref{fig:myfiga}--\ref{fig:myfigb}!,
\item  At the beginning of a sentence the word ``Figure'' should be
       written out in full. For instance, \verb!Figure~\ref{fig:myfig}!
       which gives, \eg `Figure~3'.
\end{Itemize}
Figures with several parts are cited as follows: `Figs. 2(a) and (b)',
and `Figs. 3(a)--(c)'.

Figures and illustrations \emph{should follow} the paragraph in which
they are first discussed.  If this is not feasible, they may be placed
on the following page (\LaTeX{}'s float mechanism takes care of this
automatically, in principle). If it is not possible to place
\emph{all} numbered figures in the text, then they should \emph{all}
be placed at the end of the paper.

An example with a cross-reference to a figure follows. The reference
is defined by the \verb!\label! command \emph{following} the
\verb!\caption! command. A cross-reference is generated with the
\verb!\ref! command on the second line. Note the cross-reference key
(\texttt{fig:myfig}) which clearly indicates that it refers to a
figure (see Section~\ref{sec:crossref} for a discussion of the importance of
using good keys).
\begin{Verbatim}[numbers=left,fontsize=\small]
\section{Section title}
In Fig.~\ref{fig:myfig} we see that ...
\begin{figure}
\centering\includegraphics[width=.9\linewidth]{myfig}
\caption{Description of my figure}
\label{fig:myfig}
\end{figure}
\end{Verbatim}

\subsection{Tables}

Tables are defined the the \texttt{table} environment.  Each table
should be centred on the page width, with a brief caption (specified
as argument of the \verb!\caption! command) \emph{preceding} the table
body.

In general, tables should be open, drawn with a double thin horizontal
line (0.4~pt) at the top and bottom, and single horizontal line
(0.4~pt) separating column headings from data.

Like figures, tables must be referenced in the text in consecutive
numerical order with \LaTeX's \verb!\ref! command. Examples of
cross-references to tables follow (commands that can be used to
generate the given text strings are shown between parentheses): `Table
5' (\eg \verb!\Table~\ref{tab:mytab}!),
`Tables 2--3' (\eg
\verb!\Tables~\ref{tab:mytaba}--\ref{tab:mytabb}!). The word `Table'
should never be abbreviated.

\subsubsection{Formatting and layout within the table}

Write the headings in sentence case but do not use full stops. Units
should be entered in parentheses on a separate line below the column
heading. (If the same unit is used throughout the table, it should be
written in parentheses on a separate line below the caption.)

Unsimilar items should be aligned on the left, whereas similar items
should be aligned on the operator or decimal point. All decimal points
must be preceded by a digit.

\emph{Table captions} should be brief and placed centrally
\emph{above} the table. No full stop is necessary unless the caption
is more that one sentence long, in which case full punctuation should
be used.

\emph{Notes in tables} should be designated by superscript lower-case
letters, and begun anew for each table. The superscript letter should
be placed in alphabetical order moving from left to right across the
first row and down to the last. The notes should then be listed
directly under the table.

An instance of a simple table, following the proposed rules follows.
\begin{Verbatim}[numbers=left,fontsize=\small]
Table~\ref{tab:famous} shows three famous mathematical constants.
\begin{table}
\caption{Famous constants}
\label{tab:famous}
\centering
\begin{tabular}{@{}lll@{}}\hline\hline
Symbol & Description               & Approximate value\\\hline
$e$    & base of natural logarithm & $2.7182818285$   \\
$\pi$  & ratio circle circumference to diameter &
                                     $3.1415926536$   \\
$\phi$ & golden ratio              & $1.6180339887$   \\\hline\hline
\end{tabular}
\end{table}
\end{Verbatim}
Notice the \verb!\hline! commands at the beginning and end of the
\texttt{tabular} environment to generate the double lines at the top
and the bottom of the table, as well as the single \verb!\hline! command
to separate the heading from the data. The typeset result is shown next.

Table~\ref{tab:famous} shows three famous mathematical constants.
\begin{table}[h]
\caption{Famous constants}
\label{tab:famous}
\centering
\begin{tabular}{@{}lll@{}}\hline\hline
Symbol & Description               & Approximate value\\\hline
$e$    & base of natural logarithm & $2.7182818285$   \\
$\pi$  & ratio circle circumference to diameter &
                                     $3.1415926536$   \\
$\phi$ & golden ratio              & $1.6180339887$   \\\hline\hline
\end{tabular}
\end{table}

\subsection{Bibliographical references}
\label{sec:biblioref}

References should be cited in the text using numbers within square
brackets. This is achieved with the \verb!\cite! command. Punctuation
can be used either within or outside the brackets, but the same method
should be used consistently throughout the contribution. A few examples follow.
\begin{Itemize}
\item \verb!Ref.~\cite{bib:mybiba}!  typesets, \eg `Ref.~[1]'.
\item \verb!Refs~\cite{bib:mybiba}--\cite{tab:mybibb}!, 
      typesets, \eg `Refs~[1--5].
\item At the beginning of a sentence the word ``Reference'' should be
       written out in full. For instance \verb!Reference~\cite{bib:mybibc}!
       which gives, \eg `Reference~[3]'.
\end{Itemize}

The text of the document should show the bibliographic references in
\emph{consecutive numerical order}. References in tables should be
entered in the order left to right, and top to bottom.

\subsubsection{List of references}

The list of references (the \verb!\bibitem! entries) must all be
grouped inside a \texttt{thebibliography} environment, as follows.
\begin{Verbatim}[numbers=left,fontsize=\small]
\section{Section title}
Work by Einstein~\cite{bib:einstein} as well as that in
Ref.~\cite{bib:gravitation} explain the theory of relativity.
...
\begin{thebibliography}
\bibitem{bib:einstein}} ...
\bibitem{bib:gravitation} ...
  ...
\end{thebibliography}
\end{Verbatim}
As shown, each \verb!\bibitem! is identified with a reference key
(\texttt{bib:einstein}, \texttt{bib:gravitation}, \etc), which allows
one to refer to the relevant bibliography entry with a
\verb|\cite{bib:einstein}| command in the text. The information about
each entry can be specified in a separate file, \eg
\texttt{mybib.bbl}, which can be read by the \texttt{bibtex} program
for generating the list of references (see a \LaTeX{} manual or
Chapter~12 of Ref.~\cite{bib:mittelbach2004} for details).

The order of the \verb!\bibitem! elements must ensure that the
bibliographical references in the body of the text appear in
consecutive numerical order.  Care should in particular be taken with
citation commands inside floating elements (\ie figures and tables
defined by the \texttt{figure} and \texttt{table} environments), since
their position in the final typeset output can move when the input
source is modified.

Unless you are near the bottom of the last page of text, do \emph{not}
start a new page for the list of references, but continue on the same
page. Note that in the list of references it is unnecessary to state
the title of an article or chapter in proceedings or in a collection
of papers unless a page number cannot be quoted, \eg for forthcoming
publications.

For abbreviations of names of journals quoted in the references, see
the \emph{Journal abbreviations} entry available from the Web page at
the URL \url{http://cern.ch/DTP/dtpgrammar.htm}\footnote
%    
    {The Reviews of Modern Physics site: \url{http://rmp.aps.org/info/manprep.html}
     also has a list, see \emph{Appendix B: Journal Title Abbreviations}.}
%
The entry \emph{Citation of references} on the same DTP Web page shows
more details on how to present references.

If you need to provide a bibliography, this should come after the list
of references.

\subsection{Footnotes}

Footnotes are to be avoided. If absolutely necessary, they should be
brief, and placed at the bottom of the page on which they are referred
to. Take care when citing references in the footnotes to ensure that
these are correctly numbered.

\subsection{Referencing structural elements}
\label{sec:crossref}

In this section we give a general overview of \LaTeX's reference
mechanism which makes it easy to reference structural elements. First
a \verb!\label! command, with a unique \emph{key} as its argument to
identify the structural element in question, must be placed in the
source, as follows.

\begin{Itemize}
\item For sectioning commands, such as \verb!\section!,
      \verb!\subsection!, \verb!\subsubsection!, the \verb!\label! command
      must \emph{follow} it.
\item Inside \texttt{figure} and \texttt{table} environments, the 
      \verb!\label! command must be placed \emph{after} the \verb!\caption!
      command.
\item Inside an \texttt{equation} environment the \verb!\label! command
      can be placed anywhere.
\item Inside an \texttt{eqnarray} environment, the \verb!\label! command
      can be used to identify each line, so that it must be placed
      \emph{before} each end-of-line \verb!\\!. If for a given line no
      line-number has to be produced, a \verb!\nonumber! command should
      be used.
\item Inside an \texttt{enumerate} environment a \verb!\label! command
      can be associated with each \verb!\item! command.
\item Inside a \texttt{footnote} a \verb!\label! command can be placed
      anywhere.
\end{Itemize}
As seen in all the examples in this document, for reasons of clarity
it is best to place the \verb!\label! command immediately
\emph{following} the element if refers to (rather than inserting it
inside its contents).

From any place in the document one can refer to a structural element
identified with a \verb!\label! command with the help of a \verb!\ref!
command. An example follows.
\begin{Verbatim}[numbers=left,fontsize=\small]
\section{My first section}\label{sec:first}
Figure~\ref{fig:fdesc} in Section~\ref{sec:second} shows \dots
\begin{table}
\caption{table caption text}\label{tab:tdesc}
...
\end{table}
\begin{equation}
\exp{i\pi}+1=0~.\label{eq:euler}
\end{equation}
\section{My second section}\label{sec:second}
Section~\ref{sec:first} contains Table~\ref{tab:tdesc} and Eq.~\ref{eq:euler}...
\begin{figure}
\centering\includegraphics[...]{...}
\caption{Text of figure caption}
\label{fig:fdesc}
\end{figure}
\end{Verbatim}
To easily differentiate between references to the various structural
elements it is good practice to start the key with a few characters
identifying it (\eg \texttt{sec:} for a sectioning command, such as
\verb!\section!, \verb!\subsection!, \etc, \texttt{fig:} for figures,
\texttt{tab:} for tables, and \texttt{eq:} for equations, including
\texttt{eqnarray} environments (which should have a \verb!\label!
command placed before the \verb!\\! if you want to identify the line
in question). The second part of the key should identify the
particular element clearly, \eg use of a mnemonic component, such as
\texttt{eq:euler} in the example of the equation reference
above. Avoid using keys with only digits, such as \texttt{f1},
\texttt{f2}, \etc, since, if for any reason structural elements are
eliminated or change position in the source, confusion can result.

\subsection{Appendices}

Each appendix should be laid out as the sections in the text.
Appendices are labelled alphabetically and should be referred to as
Appendix A, Appendices A--C, \etc Equations, figures and tables should
be quoted as Eq.~(A.1), Fig.~B.2, Table~C.4, \etc

\subsection{Acknowledgements}

If required, acknowledgements should appear as an unnumbered
subsection immediately before the references section.

\section{Spelling and grammar}

For more information on English grammar rules and commonly misused
words and expressions (including a guide to avoiding `franglais'),
please see the files available from the DTP Web pages (URL 
\url{http://cern.ch/DTP/dtpgrammar.htm}).

\subsection{Spelling}

CERN uses British English spelling, and `-ize' rather that
`-ise'. Here we provide a few examples for guidance:
\begin{flushleft}
\begin{tabularx}{\linewidth}{@{}lX}
-il:     & fulfil (not fulfill) \\
-re:     & centre (not center) \\
-our:    & colour (not color) \\ 
-gue:    & catalogue (not catalog, but analog is used in electronics) \\ 
-mme:    & programme (not program, unless referring to a computer
           program) \\ 
-ell-:   & labelled (not labeled) \\ 
-ce/-se: & licence (noun), license (verb), practice (noun), practise (verb) \\ 
-ize:    & organization, authorize. \\
         & Exceptions to this rule include advise, comprise, compromise, 
           concise, demise, devise, enterprise, exercise, improvise, 
           incise, precise, revise, supervise, surmise, surprise, televise.\\
\end{tabularx}
\end{flushleft}

\subsection{Punctuation}

\subsubsection{Hyphen}

Hyphens are used to avoid ambiguity, \ie in attributive compound
adjectives (compare `a little used car' and `a little-used car'), to
distinguish between words such as `reform' (change for the better) and
`re-form' (form again), and to separate double letters to aid
comprehension and pronunciation (\eg co-operate).

Hyphens are also used if a prefix or suffix is added to a proper noun,
symbol, or numeral, and in fractions: \eg non-Fermi, 12-fold,
three-quarters.

\subsubsection{En and em dashes}

En dashes are used to mean `and' (\eg space--time, Sourian--Lagrange)
or `to' (\eg 2003--2004, input--output ratio). In \LaTeX{} enter
\texttt{{-}{-}} for typeset an en dash.

An em dash is used as a parenthetical pause. Simply type with no space
on either side, \eg `the experiment\,---\,due to begin in
2007\,---\,represents a major advance...'. In \LaTeX{} enter
\texttt{{-}{-}{-}} to typeset an em dash.

\subsubsection{Quotation marks}

Double for true quotations, single for anything else. Single within
double for a quotation within a quotation. Our preferred method of
punctuation around quotation marks is to place punctuation marks
outside the quotation marks, to avoid any ambiguity: Oxford has been
called a `Home of lost causes'. Shakespeare wrote ``To be or not to be''.


\subsubsection{Apostrophe}

Do not use in plural acronyms (\eg JFETs), decades (1990s).  Do use
in plural Greek letters and symbols (\eg $\pi$'s).

\subsubsection{Colon, semi-colon, exclamation mark, question mark}

Please note that in English these punctuation marks do not require a
space before them.

\subsubsection{Punctuation in lists}

In a series of three or more terms, use a comma (sometimes called the
serial comma) before the final `and' or `or' (\eg gold, silver, or
copper coating). In a run-on list, do not introduce a punctuation mark
between the main verb and the rest of the sentence.  Avoid the use of
bullet points.

For a displayed list there are two options:

\begin{itemize}
\item[i)]  finish the introductory sentence with a colon, start the
           first item of the list with a lower-case letter, finish it
           with a semi-colon, and do the same for all items until the
           last, where a full stop is placed at the end of the text
           (as here);
\item[ii)] finish the introductory sentence with a full stop, start
           the first item with a capital letter and finish it with a
           full stop, and the same for the remaining items.
\end{itemize}

\subsection{Capitalization}

Capitalize adjectives and nouns formed from proper names,
\eg Gaussian.  Exceptions to this rule include units of measure
(amperes), particles (fermion), elements (einsteinium), and minerals
(fosterite) derived from names.  Capitalize only the name in
Avogadro's number, Debye temperature, Ohm's law, Bohr radius.

Never capitalize lower-case symbols or abbreviations. When referring
to article, paper, or report, column, sample, counter, curve, or type,
do not capitalize.

Do capitalize Theorem I, Lemma 2, Corollary 3, \etc

\subsubsection{Acronyms}

In the first instance, spell out the acronym using capital letters for
each letter used in the acronym, and provide the acronym in
parentheses, \eg Quark--Gluon Plasma (QGP).

\subsection{Numbers, symbols and units}
\label{sec:numberssymbolsunits}

Spell out numbers 1 to 9 unless they are followed by a unit or are
part of a series containing the number 10 or higher (as here); numbers
are always in roman type. Numbers should always be written out at the
beginning of a sentence.  Symbols of variables (\ie anything that can
be replaced by a number) should be typed in \emph{italics}.

In scientific texts the printed form of a symbol often implies a
meaning which is not easily captured by generic markup.  Therefore
authors using some form of generic coding (such as \LaTeX) need
to know about typographical conventions.  The following is a brief
summary of the most important rules for composing scientific
texts\cite{bib:iso31,bib:iupap,bib:bipm,bib:nist}.
 
The most important rule is \emph{consistency}: a symbol should always
be the same, whether it appears in a formula or in the text, on the
main line or as a superscript or subscript.  This means that in
\LaTeX, once you have used a symbol inside mathematics mode (`\$'),
always use it inside mathematics mode.  Inside math mode, \LaTeX{} by
default prints characters in \emph{italics}.

The following families of symbols must \emph{always} to be typeset in
roman. With \LaTeX{} roman type in maths mode can be typeset by the
\verb!\mathrm! or \verb!\text! commands.

\begin{Itemize}
\item \emph{Units}, such as g, cm, s, keV.  Note that physical
      constants are usually in italics, so units involving constants
      are mixed roman-italics, \eg GeV/\emph{c} (where the \emph{c} is
      italic because it symbolizes the speed of light, a constant).
\item \emph{Particle names}, for example p, K, q, g.
\item \emph{Chemical element names}, for example Ne, O, Cu, Fe.
\item \emph{Standard mathematical functions} (sin, det, cos, tan, Re,
      Im, \etc).  Use the built-in \LaTeX\ functions for these
      (\verb+\sin+, etc.).
\item \emph{Numbers}.
\item \emph{Names of waves or states} (p-wave) and covariant couplings
      (A for axial, V for vector), names of monopoles (E for electric,
      M for magnetic).
\item \emph{Abbreviations} that are initials of bits of words (exp, for
      experimental; min, for minimum).
\item The `d' operator in integrands (\eg d\emph{p}).
\end{Itemize}
 
By applying these rules consistently, the reader will be able to
understand at first glance the precise meaning of the symbol. The
importance of the distinction between typeface (roman and italic) for
the same symbol can be seen in Table~\ref{tab:symbols}.
\begin{table}[h]
\caption[]{Examples of the importance of using the correct typeface for
symbols, variables and constants}
\label{tab:symbols}
\centering
\begin{tabular}{@{}lp{50mm}@{\qquad}lp{50mm}@{}}
\hline\hline
\multicolumn{2}{@{}c}{Roman typeface } &
\multicolumn{2}{c@{}}{Italic typeface} \\
\hline
A & ampere (electric unit)      & $A$ & atomic number (variable)       \\
e & electron (particle name)    & $e$ & electron charge (constant)     \\
g & gluon (particle name)       & $g$ & gravitational constant         \\
l & litre (volume unit)         & $l$ & length (variable)              \\
m & metre (length unit)         & $m$ & mass (variable)                \\
p & proton (particle name)      & $p$ & momentum (variable)            \\
q & quark (particle name)       & $q$ & electric charge (variable)     \\
s & second (time unit)          & $s$ & c.m. energy squared (variable) \\
t & tonne (weight unit)         & $t$ & time (variable)                \\
V & volt (electric unit)        & $V$ & volume (variable)              \\
Z & Z boson (particle name)     & $Z$ & atomic charge (variable)       \\
\hline\hline
\end{tabular}
\end{table}

A convenient two-page summary about which typefaces to use for symbols
in scientific documents is available from
NIST\cite{bib:nisttypefaces}.

As already pointed out, \LaTeX{} implements part of the rules
mentioned above by default in math mode (\eg variables will be in math
italic). Yet care must be taken with particle names, units, \etc
inside math formulae to ensure that they come out in roman.

Symbols for units derived from proper names are written with capital
letters (\eg coulomb, 6~C).  Write the unit out in full in cases such
as `a few centimetres'.  When using symbols insert a
\emph{non-breaking space} between the number and the unit, unless it
is $\%$ or superscript, \eg \verb!10~cm!, \verb!100~GeV!,
\verb!15~nb!, but \verb!20\%!, \verb!27$^\circ$C!.

\subsection{Predefined commands}
\label{ref:predefinedcommands}

To help implement the rules given above at CERN, \LaTeX{} packages have
been developed with predefined commands for abbreviations of units,
element names, and elementary particle names.

The \texttt{cernchemsym} package\footnote
%
   {\url{http://cern.ch/XML/textproc/cernchemsym.pdf}.}
%
defines the \verb!\Isotope! command (\eg \verb!\Isotope[234][92]{U}!
typesets \Isotope[234][92]{U}. The package also defines commands for
all atomic elements, which all start with \verb!\E!, \eg \verb!\EH!
for hydrogen (\EH), \verb!\EAl! for aluminium (\EAl), \etc

The \texttt{cernunits} package\footnote
%
   {\url{http://cern.ch/XML/textproc/cernunits.pdf}.}
%
defines a set of commands, all starting with
\verb!\U!. Table~\ref{tab:predef} provides a few examples of HEP
energy and momenta. Using these commands, the examples in the last
paragraph of Section~\ref{sec:numberssymbolsunits} could be entered as
\verb!\Unit{10}{cm}! or \verb!$10\Ucm$!, \verb!100\UGeV!, \etc, where
these commands work both in text or math mode.
\begin{table}[h]
\caption[]{Examples of predefined commands in the CERN package \texttt{cernunits}}
\label{tab:predef}
\centering\small
\begin{tabular}{@{}>{\ttfamily}ll@{}}
\hline\hline
{\rmfamily Commands available in math and text mode}
                                          & Result as printed   \\\hline
\verb!\Unit{3}{Tm} and $\Unit{3}{Tm}$!    & \Unit{3}{Tm} and $\Unit{3}{Tm}$  \\
\verb!\Unit{1}{PeV} and $\Unit{1}{PeV}$!  & \Unit{1}{PeV} and $\Unit{1}{PeV}$\\
\verb|3\UeV{} and $3\UeV$|                & 3\UeV{} and $3\UeV$   \\
\verb|3\UkeV{} and $3\UkeV$|              & 3\UkeV{} and $3\UkeV$ \\
%\verb|3\UMeV{} and $3\UMeV$|              & 3\UMeV{} and $3\UMeV$ \\
%\verb|3\UGeV{} and $3\UGeV$|              & 3\UGeV{} and $3\UGeV$ \\
%\verb|3\UTeV{} and $3\UTeV$|              & 3\UTeV{} and $3\UTeV$ \\
\verb|3\UPeV{} and $3\UPeV$|              & 3\UPeV{} and $3\UPeV$ \\
\verb|3\UGeVc{} and $3\UGeVc$|            & 3\UGeVc{} and $3\UGeVc$ \\
\verb|3\UGeVcc{} and $3\UGeVcc$|          & 3\UGeVcc{} and $3\UGeVcc$ \\
\hline\hline
\end{tabular}
\end{table}

The \verb!\U...!  commands typeset a non-breaking space preceding the
unit.  Each of these commands has a partner ending in `\texttt{Z}',
which omits this space, \eg \verb!100~\UGeVZ!. These variants can be
useful when combining several units.

Let us look as a few more examples: lenghts and distances (\eg
\verb!\Ufm! gives \Ufm{} for femtometre, while \verb!\Uum! gives
\Uum{} for micrometre), mass, force, energy, power and pressure (\eg
\verb!10\Ukg! gives 10\Ukg, \verb!5\UW[G]!  gives 5\UW[G],
\verb!1013\UPa[h]! gives 1013\UPa[h], the standard atmospheric
pressure), cross-sections (\verb!\Ufb! gives \Ufb{} for fermobarn),
time and frequencies (\eg, \verb!50\Ups!  gives 50\Ufb{} and
\verb!2\UGHz! gives 2\UGHz), magnetic and electric units (\eg
\verb!$1\UG=10\sp{-4}\UT$! will typeset $1\UG=10\sp{-4}\UT$),
temperature (\eg \verb!$0\UDC=273.15\UK$! gives $0\UDC=273.15\UK$),
and human health-related units (\eg \verb!10\USv! gives 10\USv{} for
ten sievert, while 50\Umrad for 50 millirad can be entered as
\verb!50\Urad[m]!  or \verb!50\Umrad!).

The \texttt{heppennames2} package\footnote
%
   {\mbox{\url{http://cern.ch/XML/textproc/heppennames2.pdf} and 
    \url{http://cern.ch/xml/pennames/pennameswww.xhtml}.}}
%
defines the \emph{Particle Entity Notation} scheme. It assigns a
unique name, starting with \verb!\P!  to all known particles in the
\emph{Particle Data Group} listing. By using this scheme the
typographic correctness of the printed symbol in \LaTeX{} (roman type
in all circumstances) is guaranteed. For example, \verb!\Pem! codes
for the electron (\Pem), and \verb!\Pep! codes for the positron (\Pep)
in both text and math mode.  A more complex example is \verb!\Pg! for
the gluon (\Pg) and \verb!\PGg! for the gamma particle (the photon
\PGg). The tau neutrino \PGnGt{} and its antiparticle \PAGnGt{} are
coded as \verb!\PGnGt! and \verb!\PAGnGt!, respectively.

%\newpage
\appendix

\section{The \texttt{cernphprep} class file}

At CERN the PH preprint files are automatically made available on AFS
(Linux) by using the latest \TeX{} installation\footnote
%
   {See \url{http://cern.ch/XML/textproc/texlivelinux2009.html}.}.
%
Example files of PH preprints are also available\footnote
%
   {See \url{http://cern.ch/XML/textproc/cernphprep.html}.}.
%
The \LaTeX{} source of the present document (\texttt{cernphprep.tex}),
as well as its typeset result (\texttt{cernphprep.pdf}) are available
there as well.

\section{The \texttt{amsmath} extensions}
\label{app:amsmath}

Recent developments around \LaTeX{}, often sponsored and driven by
scientific publishers, have ensured that structural elements and math
constructs in particular are easily delimited and their sense made
clear. The American Mathematical Society have done a great job with
their \texttt{amsmath} extension, which is loaded by default in the
\texttt{cernphprep.cls} class. Therefore, we would like our authors to
use the conventions and \LaTeX{} constructs of that package, rather
than the pure \TeX{} primitives. This is all the more important as we
want to take full advantage of the possibilities of hypertext and the
Web, where \TeX{} primitives, because they are not clearly delimited,
are much more difficult to handle (see Section~\ref{ref:predefinedcommands}).

Below we give a few often-occurring mathematics constructs, first in
their \TeX{} form (to be avoided), their \LaTeX{} equivalent (to be
preferred), and the typeset result.

\begin{itemize}
\item \verb!\over! to be replaced by \verb!\frac!.
\begin{description}
\item[\TeX{}] \verb!${Z^\nu_{i}\over 2 \Lambda_{\rm ext}}$!
\item[\LaTeX] \verb!$\frac{Z^\nu_{i}}{2 \Lambda_{\text{ext}}}$!%
   \hfill\raisebox{1em}[0mm][0mm]{%
       $\displaystyle\frac{Z^\nu_{i}}{2 \Lambda_{\text{ext}}}$}
%\item[Result] $\displaystyle\frac{Z^\nu_{i}}{2 \Lambda_{\text{ext}}}$
\end{description}
\item \verb!\choose! to be replaced by \verb!\binom!.
\begin{description}
\item[\TeX{}] \verb!${m + k \ choose k}$!
\item[\LaTeX] \verb!$\binom{m + k}{k}$!%
   \hfill\raisebox{1em}[0mm][0mm]{%
       $\displaystyle\binom{m + k}{k}$}
%\item[Result] $\binom{m + k}{k}$
\end{description}
\item \verb!\matrix! family to be replaced by corresponding environments.
\begin{description}
\item[\TeX{}] \verb!$\matrix{a & b \cr c & d\cr}$!
\item[\LaTeX] \verb!$\begin{matrix}a & b\\c & d\end{matrix}$!%
   \hfill\raisebox{1em}[0mm][0mm]{%
       $\displaystyle\begin{matrix}a & b\\c & d\end{matrix}$}
%\item[Result] $\begin{matrix}a & b\\c & d\end{matrix}$
\item[\TeX{}] \verb!$\pmatrix{a & b \cr c & d\cr}$!
\item[\LaTeX] \verb!$\begin{pmatrix}a & b\\c & d\end{pmatrix}$!%
   \hfill\raisebox{1em}[0mm][0mm]{%
       $\displaystyle\begin{pmatrix}a & b\\c & d\end{pmatrix}$}
%\item[Result] $\begin{pmatrix}a & b\\c & d\end{pmatrix}$
\item[\TeX{}] \verb!$\bmatrix{a & b \cr c & d\cr}$!
\item[\LaTeX] \verb!$\begin{bmatrix}a & b\\c & d\end{bmatrix}$!%
   \hfill\raisebox{1em}[0mm][0mm]{%
       $\displaystyle\begin{bmatrix}a & b\\c & d\end{bmatrix}$}
%\item[Result] $\begin{bmatrix}a & b\\c & d\end{bmatrix}$
\end{description}
\item Cases construct: use the \texttt{cases} environment.
\begin{description}
\item[\LaTeX{} without \texttt{amsmath}] 
\begin{minipage}[t]{.7\linewidth}
\begin{Verbatim}[fontsize=\small]
$$
{\rm curvature~} R = {6K\over a^2(t)} \quad
\left\{\begin{array}{ll}
K=-1\quad&{\rm open}\\
K=0      &{\rm flat}\\
K=+1     &{\rm closed}
\end{array}\right.
$$
\end{Verbatim}
\end{minipage}
\item[\LaTeX{} with \texttt{amsmath}]
\begin{minipage}[t]{.7\linewidth}
\begin{Verbatim}[fontsize=\small]
\[
\text{curvature } R = \frac{6K}{a^2(t)} \quad
\begin{cases}
K=-1\quad&\text{open}\\
K=0      &\text{flat}\\
K=+1     &\text{closed}
\end{cases}
\]
\end{Verbatim}
\end{minipage}
\item[Typeset result] 
\begin{minipage}[t]{.64\linewidth}
\[
\text{curvature } R = \frac{6K}{a^2(t)} \quad
\begin{cases}
K=-1\quad&\text{open}\\
K=0      &\text{flat}\\
K=+1     &\text{closed}
\end{cases}
\]
\end{minipage}
\end{description}
\item \texttt{amsmath} offers symbols for multiple integrals.
\begin{description}
\item[\TeX{}] \verb/$\int\!\!\int\!\!\int {\rm div}~\vec{\rm E}\,{\rm d}V/\\
              \verb/ = \int\!\!\int\vec{\rm E}\,{\rm d}\vec{\rm S}$/
\item[\LaTeX{}] \verb/$\iiint \text{div}~\vec{\mathrm{E}}\,\mathrm{d}V/\\
              \verb/ = \iint\vec{\mathrm{E}}\,\mathrm{d}\vec{\mathrm{S}}$/
\item[Typeset result] $\displaystyle\iiint \text{div}~\vec{\mathrm{E}}\,\mathrm{d}V
               = \iint\vec{\mathrm{E}}\,\mathrm{d}\vec{\mathrm{S}}$
\end{description}
\end{itemize}

The \texttt{cernphprep} class loads, amongst others, the
\texttt{amsmath} and the \texttt{amssymb} packages. These packages
define many supplementary commands and symbols that with plain \TeX{}
would have to be constructed from more basic components. See
Refs.~\cite{bib:voss2005} and \cite{bib:pakin2003}, or Chapter~8 of
Ref.~\cite{bib:mittelbach2004} for more details.

\begin{thebibliography}{99}\raggedright
\bibitem{bib:voss2005} Herbert Vo{\ss},
  \emph{Math mode}, available at the URL\newline
\url{http://www.tex.ac.uk/tex-archive/info/math/voss/mathmode/Mathmode.pdf}.

\bibitem{bib:mittelbach2004} Frank Mittelbach and Michel Goossens,
  \emph{The \LaTeX{} Companion}, 2nd ed. (Addison-Wesley, Boston,
  2004).

\bibitem{bib:iso31} International Organization for Standardization.
edition). Geneva: ISO, 1993.
\newblock \emph{ ISO Standards Handbook: Quantities and units (3rd edition)}.  
\newblock \url{http://en.wikipedia.org/wiki/ISO_31} and references therein.

\bibitem{bib:iupap}
International Union of pure and applied Physics.
\newblock \emph{Symbols, Units, Nomenclature and fundamental Constants
in Physics}.
\newblock Physica, 146A:1--67, 1987.

\bibitem{bib:bipm}
Bureau International des Poids et Mesures (BIPM).
\newblock \emph{The international system of units}, especially \S~5, pp.~130--135.
\newblock \url{http://www.bipm.org/utils/common/pdf/si_brochure_8.pdf}.

\bibitem{bib:nist} National Institute of Standards and Technology
(NIST).  \newblock \emph{Guide for the Use of the International System
of Units (SI)}, especially \S~10, pp.~32--38.\newline
\newblock \url{http://physics.nist.gov/cuu/pdf/sp811.pdf}

\bibitem{bib:nisttypefaces}
National Institute of Standards and Technology (NIST).
\newblock \emph{Typefaces for Symbols in Scientific Documents}.
\newblock \url{http://physics.nist.gov/Document/typefaces.pdf}.

\bibitem{bib:pakin2003} Scatt Pakin,
  \emph{The Comprehensive \LaTeX{} Symbol List}, available at the URL\newline
\url{http://www.ctan.org/tex-archive/info/symbols/comprehensive/symbols-a4.pdf}.
 
\end{thebibliography}
\end{document}

